\begin{thebibliography}{00} % using Chicago 18th shortened from Zotero

%%% Additions
Memet, Edvin, Feodor Hilitski, Margaret A Morris, Walter J Schwenger, Zvonimir Dogic, and L Mahadevan. “Microtubules Soften Due to Cross-Sectional Flattening.” eLife 7 (June 2018): e34695. https://doi.org/10.7554/eLife.34695.

Gittes, F, B Mickey, J Nettleton, and J Howard. “Flexural Rigidity of Microtubules and Actin Filaments Measured from Thermal Fluctuations in Shape.” Journal of Cell Biology 120, no. 4 (1993): 923–34. https://doi.org/10.1083/jcb.120.4.923.



%%% Microtubules in general
\bibitem[Nogales, 2025]{rev_nogales}
Nogales, Eva. Structural Insights into Microtubule Function. 2025. % review, general microtubules. Roles in cell shape, transport, motility, division.

\bibitem[Agrawal et al., 2022]{rev_transport}
Agrawal, Anamika, Zubenelgenubi C. Scott, and Elena F. Koslover. “Morphology and Transport in Eukaryotic Cells.” Annual Review of Biophysics 51, no. Volume 51, 2022 (2022): 247–66. https://doi.org/10.1146/annurev-biophys-111121-103956. % review, transport in cells

\bibitem[Schmidt and Stehbens, 2024]{rev_migration}
Schmidt, Christanny J., and Samantha J. Stehbens. “Microtubule Control of Migration: Coordination in Confinement.” Current Opinion in Cell Biology 86 (February 2024): 102289. https://doi.org/10.1016/j.ceb.2023.102289. % review

\bibitem[]{}
Gudimchuk, Nikita B., and Veronika V. Alexandrova. “Measuring and Modeling Forces Generated by Microtubules.” Biophysical Reviews 15, no. 5 (2023): 1095–110. https://doi.org/10.1007/s12551-023-01161-7. % review, force generation by microtubules

\bibitem[]{}
Sulerud, Taylor, Abdullah Bashar Sami, Guihe Li, April Kloxin, John Oakey, and Jesse Gatlin. “Microtubule-Dependent Pushing Forces Contribute to Long-Distance Aster Movement and Centration in Xenopus Laevis Egg Extracts.” Molecular Biology of the Cell 31, no. 25 (2020): 2791–802. https://doi.org/10.1091/mbc.E20-01-0088. % primary, cell division centering based on microtubule based pushing mechanisms

\bibitem[]{}
Rodionov, V I, F K Gyoeva, E Tanaka, A D Bershadsky, J M Vasiliev, and V I Gelfand. “Microtubule-Dependent Control of Cell Shape and Pseudopodial Activity Is Inhibited by the Antibody to Kinesin Motor Domain.” The Journal of Cell Biology 123, no. 6 (1993): 1811–20. https://doi.org/10.1083/jcb.123.6.1811. % primary, cell shape also requires motors

\bibitem[]{}
Behnke, O. “A Comparative Study of Microtubules of Disk-Shaped Blood Cells.” Journal of Ultrastructure Research 31, nos. 1–2 (1970): 61–75. https://doi.org/10.1016/S0022-5320(70)90145-0. % primary, marginal band required for cell shape in SOME cell types, including platelets

\bibitem[]{}
Italiano, Joseph E., Jr, Wolfgang Bergmeier, Sanjay Tiwari, et al. “Mechanisms and Implications of Platelet Discoid Shape.” Blood 101, no. 12 (2003): 4789–96. https://doi.org/10.1182/blood-2002-11-3491. % primary

\bibitem[]{}
Sadoul, K. “New Explanations for Old Observations: Marginal Band Coiling during Platelet Activation.” Journal of Thrombosis and Haemostasis 13, no. 3 (2015): 333–46. https://doi.org/10.1111/jth.12819. % review, microtubules in platelet shape




%%% Tubulin Code
\bibitem[]{}
Janke, Carsten, and Maria M. Magiera. “The Tubulin Code and Its Role in Controlling Microtubule Properties and Functions.” Nature Reviews Molecular Cell Biology 21, no. 6 (2020): 307–26. https://doi.org/10.1038/s41580-020-0214-3. % review, overall tubulin code

\bibitem[]{}
Gasic, Ivana. “Regulation of Tubulin Gene Expression: From Isotype Identity to Functional Specialization.” Frontiers in Cell and Developmental Biology 10 (May 2022). https://doi.org/10.3389/fcell.2022.898076. % review, functional specialization from different isotypes **NOT READ**

\bibitem[]{}
Janke, Carsten, and Jeannette Chloë Bulinski. “Post-Translational Regulation of the Microtubule Cytoskeleton: Mechanisms and Functions.” Nature Reviews Molecular Cell Biology 12, no. 12 (2011): 773–86. https://doi.org/10.1038/nrm3227. % review, functional specialization from different PTMS **NOT READ**


%%% Mechanical Properties
\bibitem[Ye et al., 2025]{Celeg_lumen}
Ye, Yucheng, Zheng Hao, Jingyi Luo, et al. “Tubulin Isotypes of C. Elegans Harness the Mechanosensitivity of the Lattice for Microtubule Luminal Accessibility.” Nature Physics, ahead of print (2025). https://doi.org/10.1038/s41567-025-02983-w. % primary, tubulin isotypes and mechanics as well as protofilament interactions during bending for different isotypes

\bibitem[]{}
Hawkins, Taviare, Matthew Mirigian, M. Selcuk Yasar, and Jennifer L. Ross. “Mechanics of Microtubules.” Journal of Biomechanics 43, no. 1 (2010): 23–30. https://doi.org/10.1016/j.jbiomech.2009.09.005. % review, differences in measurements of mechanical properties

\bibitem[Portran et al., 2017]{acetyl_aging}
Portran, Didier, Laura Schaedel, Zhenjie Xu, Manuel Théry, and Maxence V. Nachury. “Tubulin Acetylation Protects Long-Lived Microtubules against Mechanical Ageing.” Nature Cell Biology 19, no. 4 (2017): 391–98. https://doi.org/10.1038/ncb3481. % primary, acetylation as a tubulin PTM impacts mechanical properties AND change in bending with repeated flow cycles

\bibitem[]{}
Xu, Zhenjie, Laura Schaedel, Didier Portran, et al. “Microtubules Acquire Resistance from Mechanical Breakage through Intralumenal Acetylation.” Science 356, no. 6335 (2017): 328–32. https://doi.org/10.1126/science.aai8764. % primary, acetylation as a tubulin PTM impacts mechanical properties

\bibitem[]{}
Pampaloni, Francesco, Gianluca Lattanzi, Alexandr Jonáš, Thomas Surrey, Erwin Frey, and Ernst-Ludwig Florin. “Thermal Fluctuations of Grafted Microtubules Provide Evidence of a Length-Dependent Persistence Length.” Proceedings of the National Academy of Sciences 103, no. 27 (2006): 10248–53. https://doi.org/10.1073/pnas.0603931103. % primary, length dependent persistence length



%%% Mechanical Roles
\bibitem[]{}
Romet-Lemonne, Guillaume, Cécile Leduc, Antoine Jégou, and Hugo Wioland. “Mechanics of Single Cytoskeletal Filaments.” Annual Review of Biophysics 54, no. 1 (2025): 303–27. https://doi.org/10.1146/annurev-biophys-030722-120914. % review, mechanics of microtubules (focus on mechanical properties rather than functional roles of microtubule mechanics)

\bibitem[]{}
Matis, Maja. “The Mechanical Role of Microtubules in Tissue Remodeling.” BioEssays 42, no. 5 (2020): 1900244. https://doi.org/10.1002/bies.201900244. % review, microtubules in cellular trafficking and signaling during tissue formation as well as direct mechanical roles **NOT READ**

\bibitem[]{}
Garcin, Clare, and Anne Straube. “Microtubules in Cell Migration.” Essays in Biochemistry 63, no. 5 (2019): 509–20. https://doi.org/10.1042/EBC20190016. % review, microtubule roles in directed cell migration (as tracks for intracellular transport, force generation, compressive elements to support protrusions, and signaling) **NOT READ**


%%% Health examples
\bibitem[Uchida et al., 2022]{cardmy_review}
Uchida, Keita, Emily A. Scarborough, and Benjamin L. Prosser. “Cardiomyocyte Microtubules: Control of Mechanics, Transport, and Remodeling.” Annual Review of Physiology 84, no. Volume 84, 2022 (2022): 257–83. https://doi.org/10.1146/annurev-physiol-062421-040656. % review, microtubules in cardiomyocyte contractility

\bibitem[Caporizzo and Prosser, 2022]{card_HF}
Caporizzo, Matthew A., and Benjamin L. Prosser. “The Microtubule Cytoskeleton in Cardiac Mechanics and Heart Failure.” Nature Reviews Cardiology 19, no. 6 (2022): 364–78. https://doi.org/10.1038/s41569-022-00692-y. % review, mechanical modulation of microtubules proposed as potential therapeutic target in heart failure


%%% Microtubule networks
\bibitem[Corominas-Murtra and Petridou, 2021]{visco_review}
Corominas-Murtra, Bernat, and Nicoletta I. Petridou. “Viscoelastic Networks: Forming Cells and Tissues.” Frontiers in Physics 9 (2021). https://doi.org/10.3389/fphy.2021.666916. % review, actin and vimentin at single filament level show nonlinear increase in shear modulus at different strain amplitudes, an dmore appaent at network level with crosslinks influencing viscoelastic behavior of network. Transient and non-covalent crosslinking interactions to form viscoelastic material whereas covalent an elastic material. **NOT READ**

\bibitem[Lin et al., 2007]{visco_MTs}
Lin, Yi-Chia, Gijsje H. Koenderink, Frederick C. MacKintosh, and David A. Weitz. “Viscoelastic Properties of Microtubule Networks.” Macromolecules 40, no. 21 (2007): 7714–20. https://doi.org/10.1021/ma070862l. % primary, linear viscoelastic properties of microtubule networks


%%% Tubulin structures
\bibitem[]{}
Alushin, Gregory M., Gabriel C. Lander, Elizabeth H. Kellogg, Rui Zhang, David Baker, and Eva Nogales. “High-Resolution Microtubule Structures Reveal the Structural Transitions in Αβ-Tubulin upon GTP Hydrolysis.” Cell 157, no. 5 (2014): 1117–29. https://doi.org/10.1016/j.cell.2014.03.053. % primary, structure and conformational changes with taxol binding and GTP hydrolysis


%%% Existing measurements
\bibitem[]{}
Mameren, Joost van, Karen C. Vermeulen, Fred Gittes, and Christoph F. Schmidt. “Leveraging Single Protein Polymers To Measure Flexural Rigidity.” The Journal of Physical Chemistry B 113, no. 12 (2009): 3837–44. https://doi.org/10.1021/jp808328a. % primary, optical trapping


%%% Existing models
\bibitem[]{}
Li, C., C.Q. Ru, and A. Mioduchowski. “Length-Dependence of Flexural Rigidity as a Result of Anisotropic Elastic Properties of Microtubules.” Biochemical and Biophysical Research Communications 349, no. 3 (2006): 1145–50. https://doi.org/10.1016/j.bbrc.2006.08.153. % primary, reported difference sin flexural rigidity due to inaccuracy of isotropic beam model for shorter microtubule lengths, and that longitudinal (rather than circumferential) Young's modulus is length-independent due to anisotropic elastic properties of microtubules

\bibitem[]{}
Liew, K.M., Ping Xiang, and L.W. Zhang. “Mechanical Properties and Characteristics of Microtubules: A Review.” Composite Structures 123 (May 2015): 98–108. https://doi.org/10.1016/j.compstruct.2014.12.020. % review

\bibitem[]{}
Wang, C. Y., C. Q. Ru, and A. Mioduchowski. “Orthotropic Elastic Shell Model for Buckling of Microtubules.” Physical Review E 74, no. 5 (2006): 052901. https://doi.org/10.1103/PhysRevE.74.052901. % primary, anisotropic elastic shell model rather than isotropic elastic shell model (similar to other paper)

\bibitem[]{}
Liew, K.M., Ping Xiang, and Yuzhou Sun. “A Continuum Mechanics Framework and a Constitutive Model for Predicting the Orthotropic Elastic Properties of Microtubules.” Composite Structures 93, no. 7 (2011): 1809–18. https://doi.org/10.1016/j.compstruct.2011.01.017. % primary


%%% Microtubule bundles
\bibitem[]{}
Soheilypour, Mohammad, Mohaddeseh Peyro, Stephen J. Peter, and Mohammad R.K. Mofrad. “Buckling Behavior of Individual and Bundled Microtubules.” Biophysical Journal 108, no. 7 (2015): 1718–26. https://doi.org/10.1016/j.bpj.2015.01.030. % primary


%%% Microtubules in cells
\bibitem[]{}
Pallavicini, Carla, Alejandro Monastra, Nicolás González Bardeci, Diana Wetzler, Valeria Levi, and Luciana Bruno. “Characterization of Microtubule Buckling in Living Cells.” European Biophysics Journal 46, no. 6 (2017): 581–94. https://doi.org/10.1007/s00249-017-1207-9. % primary

\bibitem[]{}
Brangwynne, Clifford P., Frederick C. MacKintosh, Sanjay Kumar, et al. “Microtubules Can Bear Enhanced Compressive Loads in Living Cells Because of Lateral Reinforcement.” Journal of Cell Biology 173, no. 5 (2006): 733–41. https://doi.org/10.1083/jcb.200601060. % primary

\bibitem[]{}
Li, Teng. “A Mechanics Model of Microtubule Buckling in Living Cells.” Journal of Biomechanics 41, no. 8 (2008): 1722–29. https://doi.org/10.1016/j.jbiomech.2008.03.003. % primary



%%% Other
\bibitem[Okenve-Ramos et al, 2024]{neur_aging}
Okenve-Ramos, Pilar, Rory Gosling, Monika Chojnowska-Monga, et al. “Neuronal Ageing Is Promoted by the Decay of the Microtubule Cytoskeleton.” PLOS Biology 22, no. 3 (2024): e3002504. https://doi.org/10.1371/journal.pbio.3002504. 

\bibitem[]{}
Needleman, Daniel J., Miguel A. Ojeda-Lopez, Uri Raviv, et al. “Synchrotron X-Ray Diffraction Study of Microtubules Buckling and Bundling under Osmotic Stress: A Probe of Interprotofilament Interactions.” Physical Review Letters 93, no. 19 (2004): 198104. https://doi.org/10.1103/PhysRevLett.93.198104. % primary, shape change with osmotic stress and looking at interprotofilament bond strengths

\bibitem[]{}
Mickey, B., and J. Howard. “Rigidity of Microtubules Is Increased by Stabilizing Agents.” The Journal of Cell Biology 130, no. 4 (1995): 909–17. https://doi.org/10.1083/jcb.130.4.909.








\bibliographystyle{elsarticle-harv} 
% \bibliography{cas-refs}
\end{thebibliography}